\documentclass{article}

\usepackage[spanish,american]{babel}
\usepackage{fontspec}

\usepackage{fancyhdr}
\usepackage{extramarks}
\usepackage{amsmath}
\usepackage{amsthm}
\usepackage{amsfonts}
\usepackage{tikz}
\usepackage[plain]{algorithm}
\usepackage{algpseudocode}
\usepackage{enumitem}
\usepackage{physics}
\usepackage{units}

\usetikzlibrary{automata,positioning}

%
% Basic Document Settings
%

\topmargin=-0.45in
\evensidemargin=0in
\oddsidemargin=0in
\textwidth=6.5in
\textheight=9.0in
\headsep=0.25in

\linespread{1.1}

\pagestyle{fancy}
\lhead{\hmwkAuthorName}
\chead{\hmwkClass: \hmwkTitle}
\rhead{\firstxmark}
\lfoot{\lastxmark}
\cfoot{\thepage}

\renewcommand\headrulewidth{0.4pt}
\renewcommand\footrulewidth{0.4pt}

\setlength\parindent{0pt}

%
% Create Problem Sections
%

\newcommand{\enterProblemHeader}[1]{
    \nobreak\extramarks{}{Problem \arabic{#1} continued on next page\ldots}\nobreak{}
    \nobreak\extramarks{Ejercicio \arabic{#1} (continued)}{Problem \arabic{#1} continued on next page\ldots}\nobreak{}
}

\newcommand{\exitProblemHeader}[1]{
    \nobreak\extramarks{Problem \arabic{#1} (continued)}{Problem \arabic{#1} continued on next page\ldots}\nobreak{}
    \stepcounter{#1}
    \nobreak\extramarks{Ejercicio \arabic{#1}}{}\nobreak{}
}

\setcounter{secnumdepth}{0}
\newcounter{partCounter}
\newcounter{homeworkProblemCounter}
\setcounter{homeworkProblemCounter}{1}
\nobreak\extramarks{Ejercicio \arabic{homeworkProblemCounter}}{}\nobreak{}

%
% Homework Problem Environment
%
% This environment takes an optional argument. When given, it will adjust the
% problem counter. This is useful for when the problems given for your
% assignment aren't sequential. See the last 3 problems of this template for an
% example.
%
\newenvironment{homeworkProblem}[1][-1]{
    \ifnum#1>0
        \setcounter{homeworkProblemCounter}{#1}
    \fi
    \section{Ejercicio \arabic{homeworkProblemCounter}}
    \setcounter{partCounter}{1}
    \enterProblemHeader{homeworkProblemCounter}
}{
    \exitProblemHeader{homeworkProblemCounter}
}

%
% Homework Details
%   - Title
%   - Due date
%   - Class
%   - Section/Time
%   - Instructor
%   - Author
%

\newcommand{\hmwkTitle}{Trabajo Práctico\ 1}
\newcommand{\hmwkDueDate}{16 de Septiembre de 2016}
\newcommand{\hmwkClass}{Computación cuántica}
\newcommand{\hmwkClassLong}{Introducción a la computación cuántica y fundamentos de lenguajes de programación}
\newcommand{\hmwkClassInstructor}{Profesor Alejandro Díaz-Caro}
\newcommand{\hmwkAuthorName}{Emilio López}

%
% Title Page
%

\title{
    \vspace{2in}
    \textmd{\textbf{\hmwkClass\footnote{Formalmente ``\hmwkClassLong''}:\ \hmwkTitle}}\\
    \normalsize\vspace{0.1in}\small{Fecha\ de\ entrega:\ \hmwkDueDate}\\
    \vspace{0.1in}\large{\textit{\hmwkClassInstructor}}
    \vspace{3in}
}

\author{\textbf{\hmwkAuthorName}}
\date{}

\renewcommand{\part}[1]{\textbf{\large Parte \Alph{partCounter}}\stepcounter{partCounter}\\}

%
% Various Helper Commands
%

% Useful for algorithms
\newcommand{\alg}[1]{\textsc{\bfseries \footnotesize #1}}

% For derivatives
\newcommand{\deriv}[1]{\frac{\mathrm{d}}{\mathrm{d}x} (#1)}

% For partial derivatives
\newcommand{\pderiv}[2]{\frac{\partial}{\partial #1} (#2)}

% Integral dx
\newcommand{\dx}{\mathrm{d}x}

% Alias for the Solution section header
\newcommand{\solution}{\vspace{1em}\textbf{\large Solución}\vspace{.5em}}

% Probability commands: Expectation, Variance, Covariance, Bias
\newcommand{\E}{\mathrm{E}}
\newcommand{\Var}{\mathrm{Var}}
\newcommand{\Cov}{\mathrm{Cov}}
\newcommand{\Bias}{\mathrm{Bias}}

\begin{document}

\maketitle

\pagebreak

\begin{homeworkProblem}[6]
    Usando notación Dirac y resultados de ejercicios anteriores,
    calcular rápidamente
    $$
    \left(\left(\begin{matrix}1 \\ 2\end{matrix} \right) \otimes
            \left(\begin{matrix}3 & 4\end{matrix} \right){}
    \right)
    \left(\begin{matrix}2 \\ 3\end{matrix} \right){}
    $$

    \solution

    Primero convertimos la operación a notación bra-ket sobre la base
    ortonormal $\{\ket{0}, \ket{1}\}$, quedando
    \[
    \bigl(\left(\ket{0} + 2\ket{1}\right) \otimes
          \left(3\bra{0} + 4\bra{1}\right)
    \bigr) \,
    (2\ket{0} + 3\ket{1})
    \]

    Posteriormente aplicamos el producto tensorial, obteniendo
    \[
    \left(3\ketbra{0}{0} + 4\ketbra{0}{1} + 6\ketbra{1}{0} +
          8\ketbra{1}{1}
    \right) \,
    (2\ket{0} + 3\ket{1})
    \]

    Luego realizamos el producto restante
    \[\begin{split}
    6\ket{0}\braket{0}{0}  &+ \;\:9\ket{0}\braket{0}{1} +
    \;\:8\ket{0}\braket{1}{0}   + 12\ket{0}\braket{1}{1} + \\
    12\ket{1}\braket{0}{0} &+ 18\ket{1}\braket{0}{1} +
    16\ket{1}\braket{1}{0}  + 24\ket{1}\braket{1}{1}
    \end{split}\]

    Finalmente, y dado que $\braket{0} = \braket{1} = 1$ y
    $\braket{0}{1} = \braket{1}{0} = 0$ por ser la base ortonormal,
    podemos simplificar dicha expresión a $18\ket{0} + 36\ket{1}$

\end{homeworkProblem}

\begin{homeworkProblem}[8]
    Probar las propiedades de los operadores adjuntos listadas justo debajo de la Definición 1.14
\end{homeworkProblem}

\begin{homeworkProblem}[10]
Considerar el operador de medición $\{\dyad{+}, \dyad{-}\}$, con
$\ket{+} = \nicefrac{1}{\sqrt{2}}\ket{0} + \nicefrac{1}{\sqrt{2}}\ket{1}$
y $\ket{-} = \nicefrac{1}{\sqrt{2}}\ket{0} - \nicefrac{1}{\sqrt{2}}\ket{1}$.
Determinar los resultados posibles (y sus probabilidades) de medir con
ese operador cada uno de los siguientes qubits:
\begin{enumerate}[label=(\alph*)]
    \item $\nicefrac{1}{3}\ket{0} + \nicefrac{\sqrt{8}}{3}\ket{1}$.
    \item $\nicefrac{1}{\sqrt{2}}\, (\ket{0} + \ket{1})$.
    \item $\ket{-}$
\end{enumerate}
\end{homeworkProblem}

\begin{homeworkProblem}[13]
¿Cuáles de los estados del ejercicio anterior están en superposición con
respecto a la base $\{\ket{+}, \ket{-} \}$ y cuáles no?
\end{homeworkProblem}

\begin{homeworkProblem}[23]
Escribir la traza del algoritmo de Deutsch para la función identidad.
\end{homeworkProblem}

\begin{homeworkProblem}[25]
En una lista de un millón de elementos distintos
\begin{enumerate}[label=(\alph*)]
    \item ¿Cuál es el número óptimo de iteraciones del algoritmo de
    Grover para buscar un elemento?
    \item ¿Cuál es la probabilidad de error con el número óptimo de
    iteraciones?
    \item ¿Cuántos pasos serían necesarios, en promedio, en el caso
    clásico?
\end{enumerate}
\end{homeworkProblem}

\begin{homeworkProblem}[26]
    En el protocolo BB84, ¿cuántos bits necesitan comparar Alice y Bob para tener $90\%$ de chances de detectar la presencia de Eve?

    \solution

    Para este análisis ignoraremos los qubits en donde Alice y
    Bob no hayan usado el mismo esquema, ya que el algoritmo los descarta.
    Por cada qubit, Eve debe elegir un esquema entre $+$ o $\times$
    para observarlo y retransmitirlo. Si lo hace aleatoriamente, tendrá
    probabilidad $\nicefrac{1}{2}$ de usar el mismo que Alice y Bob
    y no modificar el qubit en tránsito. En caso contrario, el qubit
    colapsará a otro estado al realizar la medición, y será luego medido
    erróneamente por Bob un $50\%$ de las veces. Por lo tanto, la
    probabilidad de no detectar a Eve comparando un qubit de Alice y Bob
    es de $\nicefrac{3}{4}$.

    Dado que esta probabilidad por qubit es independiente, resulta que
    con comparar 8 qubits se tiene aproximadamente un $10\%$ de
    probabilidad de no detectar a Eve $\left(\left(\nicefrac{3}{4}\right)^8 \approx 0,10\right)$,
    en otras palabras, el $90\%$ de las veces Alice y Bob podrán inferir
    la presencia de Eve y accionar de forma pertinente.
\end{homeworkProblem}

\end{document}
