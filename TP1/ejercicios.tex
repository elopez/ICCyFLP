\documentclass{article}

\usepackage[spanish,american]{babel}
\usepackage{fontspec}

\usepackage{fancyhdr}
\usepackage{extramarks}
\usepackage{amsmath}
\usepackage{amsthm}
\usepackage{amsfonts}
\usepackage{tikz}
\usepackage[plain]{algorithm}
\usepackage{algpseudocode}
\usepackage{enumitem}
\usepackage{physics}
\usepackage{units}

\usetikzlibrary{automata,positioning}

%
% Basic Document Settings
%

\topmargin=-0.45in
\evensidemargin=0in
\oddsidemargin=0in
\textwidth=6.5in
\textheight=9.0in
\headsep=0.25in

\linespread{1.1}

\pagestyle{fancy}
\lhead{\hmwkAuthorName}
\chead{\hmwkClass: \hmwkTitle}
\rhead{\firstxmark}
\lfoot{\lastxmark}
\cfoot{\thepage}

\renewcommand\headrulewidth{0.4pt}
\renewcommand\footrulewidth{0.4pt}

\setlength\parindent{0pt}

%
% Create Problem Sections
%

\newcommand{\enterProblemHeader}[1]{
    \nobreak\extramarks{}{El ejercicio \arabic{#1} continúa en la página siguiente\ldots}\nobreak{}
    \nobreak\extramarks{Ejercicio \arabic{#1} (cont.)}{El ejercicio \arabic{#1} continúa en la página siguiente\ldots}\nobreak{}
}

\newcommand{\exitProblemHeader}[1]{
    \nobreak\extramarks{Ejercicio \arabic{#1} (cont.)}{El ejercicio \arabic{#1} continúa en la página siguiente\ldots}\nobreak{}
    \stepcounter{#1}
    \nobreak\extramarks{Ejercicio \arabic{#1}}{}\nobreak{}
}

\setcounter{secnumdepth}{0}
\newcounter{partCounter}
\newcounter{homeworkProblemCounter}
\setcounter{homeworkProblemCounter}{1}
\nobreak\extramarks{Ejercicio \arabic{homeworkProblemCounter}}{}\nobreak{}

%
% Homework Problem Environment
%
% This environment takes an optional argument. When given, it will adjust the
% problem counter. This is useful for when the problems given for your
% assignment aren't sequential. See the last 3 problems of this template for an
% example.
%
\newenvironment{homeworkProblem}[1][-1]{
    \ifnum#1>0
        \setcounter{homeworkProblemCounter}{#1}
    \fi
    \section{Ejercicio \arabic{homeworkProblemCounter}}
    \setcounter{partCounter}{1}
    \enterProblemHeader{homeworkProblemCounter}
}{
    \exitProblemHeader{homeworkProblemCounter}
}

%
% Homework Details
%   - Title
%   - Due date
%   - Class
%   - Section/Time
%   - Instructor
%   - Author
%

\newcommand{\hmwkTitle}{Trabajo Práctico\ 1}
\newcommand{\hmwkDueDate}{16 de Septiembre de 2016}
\newcommand{\hmwkClass}{Computación cuántica}
\newcommand{\hmwkClassLong}{Introducción a la computación cuántica y fundamentos de lenguajes de programación}
\newcommand{\hmwkClassInstructor}{Profesor Alejandro Díaz-Caro}
\newcommand{\hmwkAuthorName}{Emilio López}

%
% Title Page
%

\title{
    \vspace{2in}
    \textmd{\textbf{\hmwkClass\footnote{Formalmente ``\hmwkClassLong''}:\ \hmwkTitle}}\\
    \normalsize\vspace{0.1in}\small{Fecha\ de\ entrega:\ \hmwkDueDate}\\
    \vspace{0.1in}\large{\textit{\hmwkClassInstructor}}
    \vspace{3in}
}

\author{\textbf{\hmwkAuthorName}}
\date{}

\renewcommand{\part}[1]{\textbf{\large Parte \Alph{partCounter}}\stepcounter{partCounter}\\}

%
% Various Helper Commands
%

% Useful for algorithms
\newcommand{\alg}[1]{\textsc{\bfseries \footnotesize #1}}

% For derivatives
\newcommand{\deriv}[1]{\frac{\mathrm{d}}{\mathrm{d}x} (#1)}

% For partial derivatives
\newcommand{\pderiv}[2]{\frac{\partial}{\partial #1} (#2)}

% Integral dx
\newcommand{\dx}{\mathrm{d}x}

% Alias for the Solution section header
\newcommand{\solution}{\vspace{1em}\textbf{\large Solución}\vspace{.5em}}

% Probability commands: Expectation, Variance, Covariance, Bias
\newcommand{\E}{\mathrm{E}}
\newcommand{\Var}{\mathrm{Var}}
\newcommand{\Cov}{\mathrm{Cov}}
\newcommand{\Bias}{\mathrm{Bias}}

\begin{document}

\maketitle

\pagebreak

\begin{homeworkProblem}[6]
    Usando notación Dirac y resultados de ejercicios anteriores,
    calcular rápidamente
    \[
    \left(\left(\begin{matrix}1 \\ 2\end{matrix} \right) \otimes
            \left(\begin{matrix}3 & 4\end{matrix} \right)
    \right)
    \left(\begin{matrix}2 \\ 3\end{matrix} \right)
    \]

    \solution

    Al reescribir la operación en notación bra-ket sobre la base
    ortonormal $\{\ket{0}, \ket{1}\}$, queda
    \[
    \bigl(\left(\ket{0} + 2\ket{1}\right) \otimes
          \left(3\bra{0} + 4\bra{1}\right)
    \bigr) \,
    (2\ket{0} + 3\ket{1})
    \]

    Posteriormente es posible aplicar el producto tensorial, obteniendo
    \[
    \left(3\ketbra{0}{0} + 4\ketbra{0}{1} + 6\ketbra{1}{0} +
          8\ketbra{1}{1}
    \right) \,
    (2\ket{0} + 3\ket{1})
    \]

    Luego se evalúa el producto vectorial restante
    \[\begin{split}
    6\ket{0}\braket{0}{0}  &+ \;\:9\ket{0}\braket{0}{1} +
    \;\:8\ket{0}\braket{1}{0}   + 12\ket{0}\braket{1}{1} + \\
    12\ket{1}\braket{0}{0} &+ 18\ket{1}\braket{0}{1} +
    16\ket{1}\braket{1}{0}  + 24\ket{1}\braket{1}{1}
    \end{split}\]

    Finalmente, y dado que $\braket{0} = \braket{1} = 1$ y
    $\braket{0}{1} = \braket{1}{0} = 0$ por ortonormalidad de la base,
    dicha expresión resulta simplificable a $18\ket{0} + 36\ket{1}$

\end{homeworkProblem}

\begin{homeworkProblem}[8]
    Probar las propiedades de los operadores adjuntos listadas justo debajo de la Definición 1.14

    \solution
    \begin{proof}[Prueba de $(A^{\dag})^{\dag} = A$]
    Sea $A \in \mathcal{M}_{n x m}$ con componentes $a_{ij}$. Luego, por
    definición de adjunto, $A^{\dag} \in \mathcal{M}_{m x n}$ tendrá
    componentes $a^{*}_{ji}$. Análogamente, las componentes de
    $(A^{\dag})^{\dag} \in \mathcal{M}_{n x m}$ serán
    $(a^{*}_{ij})^{*} = (a_{ij})$. Finalmente, por igualdad de matrices
    se puede afirmar que $(A^{\dag})^{\dag} = A$.
    \end{proof}

    \begin{proof}[Prueba de $(A + B)^{\dag} = A^{\dag} + B^{\dag}$]
    Sean $A, B \in \mathcal{M}_{n x m}$ con componentes $a_{ij}$ y
    $b_{ij}$ respectivamente. Por definición de adjunto,
    $A^{\dag}, B^{\dag} \in \mathcal{M}_{m x n}$ tienen componentes
    $a_{ji}^{*}$ y $b_{ji}^{*}$. A su vez, la suma de matrices $A + B$
    tendrá componentes $a_{ij} + b_{ij}$ y $(A + B)^{\dag}$ estará
    compuesta por $(a_{ji} + b_{ji})^{*}$. Aplicando propiedades del
    conjugado, resulta que $(a_{ji} + b_{ji})^{*} = a_{ji}^{*} + b_{ji}^{*}$,
    lo que equivale a la suma de componentes de $A^{\dag}$ y $B^{\dag}$.
    Por igualdad y suma de matrices luego vale que
    $(A + B)^{\dag} = A^{\dag} + B^{\dag}$.
    \end{proof}

    \begin{proof}[Prueba de $(\alpha A)^{\dag} = \alpha^{*} A^{\dag}$]
    Sea $A \in \mathcal{M}_{n x m}$, con componentes $\alpha\, a_{ij}$.
    Por definición de adjunto, $A^{\dag}$ está compuesta por
    $a_{ji}^{*}$. Por otro lado, $(\alpha A)^{\dag}$ tiene componentes
    $(\alpha\, a_{ji})^{*}$ y por propiedad del conjugado resulta
    $(\alpha\, a_{ji})^{*} = \alpha^{*} a_{ji}^{*}$. De esto se concluye
    por igualdad de matrices que $(\alpha A)^{\dag} = \alpha^{*} A^{\dag}$.
    \end{proof}

    \begin{proof}[Prueba de $(AB)^{\dag} = B^{\dag} A^{\dag}$]
    Sean $A, B \in \mathcal{M}_{n x m}$. Por definición de adjunto y
    propiedades del conjugado, $(AB)^{\dag} = (A^{*} B^{*})^{T}$. Luego,
    dada la propiedad de matrices $(AB)^{T} = B^{T} A^{T}$ y por
    definición de adjunto resulta que
    $(A^{*} B^{*})^{T} = (B^{*})^{T} (A^{*})^{T} = B^{\dag} A^{\dag}$.
    \end{proof}

    \begin{proof}[Prueba de $\bra{A\psi} = \bra{\psi} A^{\dag}$]
    TODO
    \end{proof}
\end{homeworkProblem}

\begin{homeworkProblem}[10]
    Considerar el operador de medición $\{\dyad{+}, \dyad{-}\}$, con
    $\ket{+} = \nicefrac{1}{\sqrt{2}}\ket{0} + \nicefrac{1}{\sqrt{2}}\ket{1}$
    y $\ket{-} = \nicefrac{1}{\sqrt{2}}\ket{0} - \nicefrac{1}{\sqrt{2}}\ket{1}$.
    Determinar los resultados posibles (y sus probabilidades) de medir con
    ese operador cada uno de los siguientes qubits:
    \begin{enumerate}[label=(\alph*)]
        \item $\nicefrac{1}{3}\ket{0} + \nicefrac{\sqrt{8}}{3}\ket{1}$.
        \item $\nicefrac{1}{\sqrt{2}}\, (\ket{0} + \ket{1})$.
        \item $\ket{-}$
    \end{enumerate}

    \solution

    Para resolver este ejercicio se puede aplicar un resultado equivalente
    al obtenido en el \emph{ejemplo 1.14} del apunte, pero en esta
    ocasión con qubits sobre la base $\{\ket{+}, \ket{-}\}$. Dado un qubit
    $\alpha\ket{+} + \beta\ket{-}$ y el operador provisto, se tiene una
    probabilidad de $|\alpha|^2$ de aplicar el proyector $\dyad{+}$ y
    evolucionar el sistema a $\frac{\alpha}{|\alpha|}\ket{+}$, y una
    probabilidad $|\beta|^2$ de aplicar el proyector $\dyad{-}$ y evolucionar
    a $\frac{\beta}{|\beta|}\ket{-}$.

    Para convertir de $\{\ket{0}, \ket{1}\}$ a $\{\ket{+}, \ket{-}\}$
    resulta útil tener la equivalencia inversa. Esta se puede obtener
    operando con las definiciones, y resulta ser
    $\ket{0} = \nicefrac{1}{\sqrt{2}}\ket{+} + \nicefrac{1}{\sqrt{2}}\ket{-}$
    y $\ket{1} = \nicefrac{1}{\sqrt{2}}\ket{+} - \nicefrac{1}{\sqrt{2}}\ket{-}$.

    \begin{enumerate}[label=(\alph*)]
        \item Reemplazando $\ket{+}$ y $\ket{-}$ se tiene
        \[\begin{aligned}
        \frac{1}{3}\ket{0} + \frac{\sqrt{8}}{3}\ket{1} &=
        \frac{1}{3}\left(\frac{1}{\sqrt{2}}\ket{+} + \frac{1}{\sqrt{2}}\ket{-}\right) + \frac{\sqrt{8}}{3}\left(\frac{1}{\sqrt{2}}\ket{+} - \frac{1}{\sqrt{2}}\ket{-}\right) \\
        &= \frac{1 + \sqrt{8}}{3\sqrt{2}}\ket{+} + \frac{1 - \sqrt{8}}{3\sqrt{2}}\ket{-} \\
        \end{aligned}\]

        Luego la probabilidad de medir $\ket{+}$ es de
        $\left(\frac{1 + \sqrt{8}}{3\sqrt{2}}\right)^{2} = \frac{9 + 2\sqrt{8}}{18}$
        y la probabilidad de medir $\ket{-}$ es de
        $\left(\frac{1 - \sqrt{8}}{3\sqrt{2}}\right)^{2} = \frac{9 - 2\sqrt{8}}{18}$.

        \item $\nicefrac{1}{\sqrt{2}}\, (\ket{0} + \ket{1}) = \nicefrac{1}{\sqrt{2}} \ket{0} + \nicefrac{1}{\sqrt{2}} \ket{1} = \ket{+}$.
        Luego se medirá $\ket{+}$ con probabilidad 1.

        \item Al aplicar el operador de medición sobre $\ket{-}$ se obtendrá
        $\ket{-}$ con probabilidad 1.
    \end{enumerate}

\end{homeworkProblem}

\begin{homeworkProblem}[13]
    ¿Cuáles de los estados del ejercicio anterior\footnote{ejercicio 12 de la práctica}
    están en superposición con respecto a la base $\{\ket{+}, \ket{-} \}$
    y cuáles no?

    \solution

    Para responder, se cambiará de base como fue explicado en el ejercicio
    10. Un qubit se dice en superposición si resulta ser una combinación
    lineal con coeficientes no nulos de los elementos de la base. En otras
    palabras, dado un qubit $\alpha\ket{+} + \beta\ket{-}$, está en
    superposición si y sólo si $|\alpha| \neq 0$ y $|\beta| \neq 0$

    \begin{enumerate}[label=(\alph*)]
        \item $\ket{0} = \nicefrac{1}{\sqrt{2}}\ket{+} + \nicefrac{1}{\sqrt{2}}\ket{-}$,
        por lo que está en superposición respecto a la base.
        \item $\ket{+}$ no está en superposición respecto a la base, ya
        que es un elemento de la misma.
        \item $\nicefrac{1}{\sqrt{2}}\left(\ket{+} + \ket{-}\right) = \nicefrac{1}{\sqrt{2}}\ket{+} + \nicefrac{1}{\sqrt{2}}\ket{-}$
        por lo que se encuentra en superposición respecto a la base.
        \item $\nicefrac{1}{\sqrt{2}}\left(\ket{+} - \ket{-}\right) = \nicefrac{1}{\sqrt{2}}\ket{+} - \nicefrac{1}{\sqrt{2}}\ket{-}$
        por lo que se encuentra en superposición respecto a la base.
        \item $\nicefrac{3}{\sqrt{2}}\ket{+} - \nicefrac{1}{\sqrt{2}}\ket{-}$
        es combinación lineal con coeficientes no nulos por lo que
        también encuentra en superposición respecto a la base.
        \item $\nicefrac{1}{\sqrt{2}}\ket{0} - \nicefrac{1}{\sqrt{2}}\ket{1} = \ket{-}$,
        elemento de la base, por lo que no se encuentra en superposición respecto a ella.
    \end{enumerate}
\end{homeworkProblem}

\begin{homeworkProblem}[23]
Escribir la traza del algoritmo de Deutsch para la función identidad.
\end{homeworkProblem}

\begin{homeworkProblem}[25]
En una lista de un millón de elementos distintos
\begin{enumerate}[label=(\alph*)]
    \item ¿Cuál es el número óptimo de iteraciones del algoritmo de
    Grover para buscar un elemento?
    \item ¿Cuál es la probabilidad de error con el número óptimo de
    iteraciones?
    \item ¿Cuántos pasos serían necesarios, en promedio, en el caso
    clásico?
\end{enumerate}
\end{homeworkProblem}

\begin{homeworkProblem}[26]
    En el protocolo BB84, ¿cuántos bits necesitan comparar Alice y Bob
    para tener $90\%$ de chances de detectar la presencia de Eve?

    \solution

    Para este análisis se ignoran los qubits en donde Alice y
    Bob no hayan usado el mismo esquema, ya que el algoritmo los descarta.
    Por cada qubit, Eve debe elegir un esquema entre $+$ o $\times$
    para observarlo y retransmitirlo. Si lo hace aleatoriamente, tendrá
    probabilidad $\nicefrac{1}{2}$ de usar el mismo esquema que Alice y
    Bob y no modificar el qubit en tránsito. En caso contrario, el qubit
    colapsará a otro estado al realizar la medición, y será luego medido
    erróneamente por Bob un $50\%$ de las veces. Por lo tanto, la
    probabilidad de no detectar a Eve comparando un qubit de Alice y Bob
    es de $\nicefrac{3}{4}$.

    Dado que esta probabilidad por qubit es independiente, resulta que
    con comparar 8 qubits se tiene aproximadamente un $10\%$ de
    probabilidad de no detectar a Eve
    $\left(\left(\nicefrac{3}{4}\right)^8 \approx 0,10\right)$;
    en otras palabras, el $90\%$ de las veces Alice y Bob podrán inferir
    la presencia de Eve y accionar de forma pertinente si comparan 8
    qubits.
\end{homeworkProblem}

\end{document}
