\documentclass{article}

\usepackage[spanish]{babel}
\usepackage{fontspec}

\usepackage{fancyhdr}
\usepackage{extramarks}
\usepackage{amsmath}
\usepackage{amsthm}
\usepackage{amssymb}
\usepackage{amsfonts}
\usepackage{tikz}
\usepackage[plain]{algorithm}
\usepackage{algpseudocode}
\usepackage{enumitem}
\usepackage{physics}
\usepackage{units}
\usepackage{comment}
\usepackage{mathtools}
\usepackage{stmaryrd}
\usepackage{bussproofs}
\DeclarePairedDelimiter\floor{\lfloor}{\rfloor}
\DeclarePairedDelimiter\ceil{\lceil}{\rceil}


\theoremstyle{definition}
\newtheorem{case}{Caso}

\makeatletter
\newcases{dlrcases}{\quad}{%
  $\m@th\displaystyle{##}$\hfil}{$\m@th\displaystyle{##}$\hfil}{\lbrace}{\rbrace}
\newcases{lrcases}{\quad}{%
  $\m@th{##}$\hfil}{{##}\hfil}{\lbrace}{\rbrace}
\makeatother

\usetikzlibrary{automata,positioning}

%
% Basic Document Settings
%

\topmargin=-0.45in
\evensidemargin=0in
\oddsidemargin=0in
\textwidth=6.5in
\textheight=9.0in
\headsep=0.25in

\linespread{1.1}

\pagestyle{fancy}
\lhead{\hmwkAuthorName}
\chead{\hmwkClass: \hmwkTitle}
\rhead{\firstxmark}
\lfoot{\lastxmark}
\cfoot{\thepage}

\renewcommand\headrulewidth{0.4pt}
\renewcommand\footrulewidth{0.4pt}

\setlength\parindent{0pt}

%
% Create Problem Sections
%

\newcommand{\enterProblemHeader}[1]{
    \nobreak\extramarks{}{El ejercicio \arabic{#1} continúa en la página siguiente\ldots}\nobreak{}
    \nobreak\extramarks{Ejercicio \arabic{#1} (cont.)}{El ejercicio \arabic{#1} continúa en la página siguiente\ldots}\nobreak{}
}

\newcommand{\exitProblemHeader}[1]{
    \nobreak\extramarks{Ejercicio \arabic{#1} (cont.)}{El ejercicio \arabic{#1} continúa en la página siguiente\ldots}\nobreak{}
    \stepcounter{#1}
    \nobreak\extramarks{Ejercicio \arabic{#1}}{}\nobreak{}
}

\setcounter{secnumdepth}{0}
\newcounter{partCounter}
\newcounter{homeworkProblemCounter}
\setcounter{homeworkProblemCounter}{1}
\nobreak\extramarks{Ejercicio \arabic{homeworkProblemCounter}}{}\nobreak{}

%
% Homework Problem Environment
%
% This environment takes an optional argument. When given, it will adjust the
% problem counter. This is useful for when the problems given for your
% assignment aren't sequential. See the last 3 problems of this template for an
% example.
%
\newenvironment{homeworkProblem}[1][-1]{
    \ifnum#1>0
        \setcounter{homeworkProblemCounter}{#1}
    \fi
    \section{Ejercicio \arabic{homeworkProblemCounter}}
    \setcounter{partCounter}{1}
    \enterProblemHeader{homeworkProblemCounter}
}{
    \exitProblemHeader{homeworkProblemCounter}
}

%
% Homework Details
%   - Title
%   - Due date
%   - Class
%   - Section/Time
%   - Instructor
%   - Author
%

\newcommand{\hmwkTitle}{Trabajo Práctico 3}
\newcommand{\hmwkDueDate}{18 de Noviembre de 2016}
\newcommand{\hmwkClass}{Computación cuántica}
\newcommand{\hmwkClassLong}{Introducción a la computación cuántica y fundamentos de lenguajes de programación}
\newcommand{\hmwkClassInstructor}{Profesor Alejandro Díaz-Caro}
\newcommand{\hmwkAuthorName}{Emilio López}

%
% Title Page
%

\title{
    \vspace{2in}
    \textmd{\textbf{\hmwkClass\footnote{Formalmente ``\hmwkClassLong''}:\ \hmwkTitle}}\\
    \normalsize\vspace{0.1in}\small{Fecha\ de\ entrega:\ \hmwkDueDate}\\
    \vspace{0.1in}\large{\textit{\hmwkClassInstructor}}
    \vspace{3in}
}

\author{\textbf{\hmwkAuthorName}}
\date{}

\renewcommand{\part}[1]{\textbf{\large Parte \Alph{partCounter}}\stepcounter{partCounter}\\}

%
% Various Helper Commands
%

% Useful for algorithms
\newcommand{\alg}[1]{\textsc{\bfseries \footnotesize #1}}

% For derivatives
\newcommand{\deriv}[1]{\frac{\mathrm{d}}{\mathrm{d}x} (#1)}

% For partial derivatives
\newcommand{\pderiv}[2]{\frac{\partial}{\partial #1} (#2)}

% Integral dx
\newcommand{\dx}{\mathrm{d}x}

% Alias for the Solution section header
\newcommand{\solution}{\vspace{1em}\textbf{\large Solución}\vspace{.5em}}

% Probability commands: Expectation, Variance, Covariance, Bias
\newcommand{\E}{\mathrm{E}}
\newcommand{\Var}{\mathrm{Var}}
\newcommand{\Cov}{\mathrm{Cov}}
\newcommand{\Bias}{\mathrm{Bias}}

\newcommand{\nat}{\mathrm{nat}}

\newcommand{\Le}[3]{\lambda #1 : #2 \;.\; #3}
\newcommand{\Me}[3]{\mu #1 : #2 \;.\; #3}

\newcommand{\Val}[2][\theta]{\left\llbracket #2 \right\rrbracket_{#1}}

\begin{document}

\maketitle

\pagebreak

\begin{homeworkProblem}[37]
    Dar la semántica operacional a grandes pasos, a pequeños pasos y la
    denotacional a los siguientes términos:

    \begin{enumerate}[label=(\alph*)]
        \setcounter{enumi}{4}
        \item $\left( \Le{x}{\nat}{\Me{x}{\nat}{x + 1}} \right) \left(\left(\Le{x}{\nat}{x}\right) 2\right)$
        \item $\Me{f}{\nat \Rightarrow \nat}{\Le{n}{\nat}{\Le{m}{\nat}{m\; ?\; 1\; :\; n \times f\; n\; (m-1)}}}$
    \end{enumerate}

    \solution

    \begin{enumerate}[label=(\alph*)]
        \setcounter{enumi}{4}
        \item \textbf{Semántica operacional a pequeños pasos}
        \[\begin{aligned}
        \left( \Le{x}{\nat}{\Me{x}{\nat}{x + 1}} \right) \left(\left(\Le{x}{\nat}{x}\right) 2\right) &= \left( {\Me{x}{\nat}{x + 1}} \right)[\left(\left(\Le{x}{\nat}{x}\right) 2\right)/x]\\
        &= \left( {\Me{x}{\nat}{x + 1}} \right)\\
        &= \left( x+1 \right)[\left(\Me{x}{\nat}{x + 1}\right)/x]\\
        &= \left(\Me{x}{\nat}{x + 1}\right)+1\\
        &= \text{... (no termina, teorema 4.43)}
        \end{aligned}\]

        \textbf{Semántica operacional a grandes pasos}

        Sea \[\begin{aligned}
        A &= \Gamma \vdash \Le{x}{\nat}{\Me{x}{\nat}{x + 1}} \hookrightarrow \langle x, \Me{x}{\nat}{x + 1}, \Gamma' \rangle\\
        B &= \Gamma', x=\langle \left(\left(\Le{x}{\nat}{x}\right) 2\right), \Gamma' \rangle \vdash \Me{x}{\nat}{x + 1} \hookrightarrow \Me{x}{\nat}{x + 1}
        \end{aligned}\]

        \begin{prooftree}
        \AxiomC{A}
        \AxiomC{B}
        \BinaryInfC{$\vdash \left(\Le{x}{\nat}{\Me{x}{\nat}{x + 1}} \right) \left(\left(\Le{x}{\nat}{x}\right) 2\right) \hookrightarrow \Me{x}{\nat}{x + 1}$}
        \end{prooftree}

        \textbf{Semántica denotacional}
        \[\begin{aligned}
        \Val{\left( \Le{x}{\nat}{\Me{x}{\nat}{x + 1}} \right) \left(\left(\Le{x}{\nat}{x}\right) 2\right)} &= \Val{\Le{x}{\nat}{\Me{x}{\nat}{x + 1}}} \Val{\left(\Le{x}{\nat}{x}\right) 2}\\
        &= \left(\Le{a}{\Val[]{nat}}{\Val[\theta,x=a]{\Me{x}{\nat}{x + 1}}}\right) \Val{\left(\Le{x}{\nat}{x}\right) 2}\\
        &= \left(\Le{a}{\mathbb{N}}{\bot_{\mathbb{N}}}\right) \Val{\left(\Le{x}{\nat}{x}\right) 2}\\
        &= \bot_{\mathbb{N}}
        \end{aligned}\]

        \item \textbf{Semántica operacional a pequeños pasos}

        Sea $Exp = \Me{f}{\nat \Rightarrow \nat}{\Le{n}{\nat}{\Le{m}{\nat}{m\; ?\; 1\; :\; n \times f\; n\; (m-1)}}}$
        \[\begin{aligned}
        Exp &= \left(\Le{m}{\nat}{m\; ?\; 1\; :\; n \times f\; n\; (m-1)}\right)[Exp/f]\\
        &= \left(\Le{m}{\nat}{m\; ?\; 1\; :\; n \times Exp\; n\; (m-1)}\right)
        \end{aligned}\]

        \textbf{Semántica operacional a grandes pasos}

        Sea \[\begin{aligned}
        Exp &= \Me{f}{\nat \Rightarrow \nat}{\Le{n}{\nat}{\Le{m}{\nat}{m\; ?\; 1\; :\; n \times f\; n\; (m-1)}}}\\
        LN &=  \Le{n}{\nat}{\Le{m}{\nat}{m\; ?\; 1\; :\; n \times f\; n\; (m-1)}}\\
        LM &=  \Le{m}{\nat}{m\; ?\; 1\; :\; n \times f\; n\; (m-1)}
        \end{aligned}\]


        \begin{prooftree}
        \AxiomC{$\Gamma,f=\langle Exp, \Gamma\rangle \vdash LN \hookrightarrow \langle n, LM, \left(\Gamma,f=\langle Exp, \Gamma\rangle\right)\rangle$}
        \UnaryInfC{$\vdash Exp \hookrightarrow \langle n, LM, \left(\Gamma,f=\langle Exp, \Gamma\rangle\right)\rangle$}
        \end{prooftree}

        \textbf{Semántica denotacional}
    \end{enumerate}
\end{homeworkProblem}

\begin{homeworkProblem}[38]
    Modificar la semántica denotacional incluyendo el elemento
    \emph{error} en todo conjunto, con el fin de detectar la división
    por 0.

    \solution

    Sea el elemento \emph{error} un habitante de todos los conjuntos,
    la semántica denotacional de PCF con detección de división por cero
    queda definida de la siguiente manera:

    \newcommand{\err}{\textrm{error}}
    \[\begin{aligned}
    \Val{n} &= n \\
    \Val{x} &= \theta(x)\\
    \Val{\Le{x}{A}{t}} &= \begin{cases}
        \err & \textrm{si } \err = \Val{t} \\
        \Le{a}{\Val[]{A}}{\Val[\theta,x=a]{t}} & \textrm{en otro caso}
        \end{cases}\\
    \Val{t\; r} &= \begin{cases}
        \err & \textrm{si } \err \in \{\Val{t}, \Val{r}\} \\
        \Val{t}\; \Val{r} & \textrm{en otro caso}
    \end{cases}\\
    %\Val{\err} &= \err\\
    \Val{t \odot r} &= \begin{cases}
        \err & \textrm{si } \err \in \{\Val{t}, \Val{r}\} \\
        \err & \textrm{si } \odot \textrm{ es la división y } \Val{r} = 0\\
        \Val{t} \odot \Val{r} & \textrm{si no, y } \Val{t}, \Val{r} \in \mathbb{N}\\
        \bot_{\mathbb{N}} & \textrm{en otro caso}
    \end{cases}\\
    \Val{t\; ?\; r : s} &= \begin{cases}
        \err & \textrm{si } \err \in \{\Val{t}, \Val{r}, \Val{s}\} \\
        \Val{r} & \textrm{si no, y } \Val{t} = 0\\
        \Val{s} & \textrm{si no, y } \Val{t} \in \mathbb{N}^{*}\\
        \bot_{\mathbb{A}} & \textrm{si no, y } \Val{t} = \bot_{\mathbb{N}} \textrm{ y A es el tipo del término}
    \end{cases}\\
    \Val{\textrm{let }x : A = r \textrm{ in } t} &= \begin{cases}
        \err & \textrm{si } \err \in \{\Val[\theta,x=\Val{r}]{t}, \Val{r}\} \\
        \Val[\theta,x=\Val{r}]{t} & \textrm{en otro caso}
    \end{cases}\\
    \Me{x}{A}{t}  &= \begin{cases}
        \err & \textrm{si } \err = \Val{t} \\
        \mathrm{FIX}(\Le{a}{\Val[]{A}}{\Val[\theta,x=a]{t}}) & \textrm{en otro caso}
    \end{cases}\\
    \end{aligned}\]

    Vale notar que esta semántica denotacional no captura todos los
    errores posibles. Por ejemplo
    \[\begin{aligned}
    \Val{\left(\Le{x}{nat}{10/x}\right) 0} &= \Val{\Le{x}{nat}{10/x}} \Val{0} \\
    &= \left(\Le{a}{\Val[]{nat}}{\Val[\theta,x=a]{10/x}}\right) 0\\
    &= \left(\Le{a}{\mathbb{N}}{\Val[\theta,x=a]{10}/\Val[\theta,x=a]{x}}\right) 0\\
    &= \left(\Le{a}{\mathbb{N}}{10/a}\right) 0\\
    \end{aligned}\]

    lo que resulta en una división por cero.
\end{homeworkProblem}

\begin{homeworkProblem}[40]
    Probar el teorema (4.45): Si $\Gamma \vdash t : A$, entonces para
    toda valuación $\theta$ válida en $\Gamma$ se tiene
    $\Val{t} \in \llbracket A \rrbracket$

    \solution

    TODO
\end{homeworkProblem}

\begin{homeworkProblem}[41]
    Probar el teorema de soundness (4.46): Si $\Gamma \vdash t : A$ y
    $t \rightarrow r$, entonces para toda valuación $\theta$ válida en
    $\Gamma$ se tiene
    $\llbracket t \rrbracket_{\theta} = \llbracket r \rrbracket_{\theta}$.

    \solution

    Para poder completar esta prueba es necesario probar de antemano el
    siguiente lema: $\Val{u[t/x]} = \Val[\theta,x=\Val{t}]{u}$. Para ello
    se procede por inducción estructural sobre u.

    \begin{proof}
    \begin{case}{$u = n$.}
    Dado que $n[t/x] = n$, se concluye que $\Val{n[t/x]} = \Val[\theta,x=\Val{t}]{n}$.
    \end{case}
    \begin{case}{$u = x$.}
    Dado que $x[t/x] = t$, es claro que $\Val{x[t/x]} = \Val{t} = \Val[\theta,x=\Val{t}]{x}$.
    \end{case}
    \begin{case}{$u = y, y \neq x$.}
    En este caso $y[t/x] = y$, luego $\Val{y[t/x]} = \Val[\theta,x=\Val{t}]{y}$.
    \end{case}
    \begin{case}{$u = \Le{y}{C}{r}$.} Sin pérdida de generalidad, asumamos que
    $y \notin FV(t)$ (caso contrario, se renombrará $y$ en $t$). Por hipótesis
    inductiva sabemos que $\Val{r[t/x]} = \Val[\theta,x=\Val{t}]{r}$. Luego
    \[\Val{\left(\Le{y}{C}{r}\right)[t/x]} = \begin{cases}
        \Val{\left(\Le{y}{C}{r}\right)} = \Le{a}{\Val[]{C}}{\Val[\theta,y=a]{r}} & \textrm{si } x = y\\
        \Val{\left(\Le{y}{C}{r[t/x]}\right)} = \Le{a}{\Val[]{C}}{\Val[\theta,y=a]{r[t/x]}} = \Le{a}{\Val[]{C}}{\Val[\theta,x=\Val{t},y=a]{r}} & \textrm{si } x \neq y
    \end{cases}\]
    Si consideramos $\theta$ del primer caso como $\theta,x=\Val[\theta]{t}$,
    hecho que no varía el entorno de la evaluación ya que $x=y$ y luego aparece
    $y=a$, resulta en ambos casos $\Le{a}{\Val[]{C}}{\Val[\theta,x=\Val{t},y=a]{r}} = \Val[\theta,x=\Val{t}]{\Le{y}{C}{r}}$.
    \end{case}
    \begin{case}{$u = r\; s$.}
    Ya que $(r\; s)[t/x] = r[t/x]\; s[t/x]$, por hipótesis de inducción
    vale que \[\Val{r[t/x]} = \Val[\theta,x=\Val{t}]{r} \;\textrm{ y }\;
    \Val{s[t/x]} = \Val[\theta,x=\Val{t}]{s}\]
    De ello resulta que
    $\Val{(r\; s)[t/x]} = \Val{r[t/x]\; s[t/x]} = \Val{r[t/x]} \Val{s[t/x]} = \Val[\theta,x=\Val{t}]{r} \Val[\theta,x=\Val{t}]{s} = \Val[\theta,x=\Val{t}]{r\; s}$
    \end{case}
    \begin{case}{$u = r \odot s$.}
    Dado que $(r \odot s)[t/x] = r[t/x] \odot s[t/x]$, por hipótesis de
    inducción vale que \[\Val{r[t/x]} = \Val[\theta,x=\Val{t}]{r} \;\textrm{ y }\;
    \Val{s[t/x]} = \Val[\theta,x=\Val{t}]{s}\]
    Luego $\Val{(r \odot s)[t/x]} = \Val{r[t/x] \odot s[t/x]} = \Val{r[t/x]} \odot \Val{s[t/x]} = \Val[\theta,x=\Val{t}]{r} \odot \Val[\theta,x=\Val{t}]{s} = \Val[\theta,x=\Val{t}]{r \odot s}$
    \end{case}
    \begin{case}{$u = r\; ?\; s : o$.}
    Dado que $(r\; ?\; s : o)[t/x] = r[t/x]\; ?\; s[t/x] : o[t/x]$, por hipótesis de
    inducción vale que \[\Val{r[t/x]} = \Val[\theta,x=\Val{t}]{r} \;\textrm{ y }\;
    \Val{s[t/x]} = \Val[\theta,x=\Val{t}]{s} \;\textrm{ y }\;
    \Val{o[t/x]} = \Val[\theta,x=\Val{t}]{o}\]
    De ello se desprende que
    \[\begin{aligned}
    \Val{(r\; ?\; s : o)[t/x]} &= \Val{r[t/x]\; ?\; s[t/x] : o[t/x]}\\ &=\begin{dlrcases}
    \Val{s[t/x]} = \Val[\theta,x=\Val{t}]{s} & \textrm{si } \Val{r[t/x]} = \Val[\theta,x=\Val{t}]{r} = 0\\
    \Val{o[t/x]} = \Val[\theta,x=\Val{t}]{o} &  \textrm{si } \Val{r[t/x]} = \Val[\theta,x=\Val{t}]{r} \in \mathbb{N}\\
    \bot_{\mathbb{A}} & \textrm{si } \Val{r[t/x]} = \Val[\theta,x=\Val{t}]{r} = \bot_{\mathbb{N}} \textrm{ y A es el tipo del término}
    \end{dlrcases} \\&= \Val[\theta,x=\Val{t}]{r\; ?\; s : o}
    \end{aligned}\]
    \end{case}
    \begin{case}{$\Me{y}{C}{r}$.}
    Sin pérdida de generalidad, asumamos que $y \notin FV(t)$
    (caso contrario, se renombrará $y$ en $t$). Por hipótesis inductiva
    sabemos que $\Val{r[t/x]} = \Val[\theta,x=\Val{t}]{r}$. Luego
    \[\Val{\left(\Me{y}{C}{r}\right)[t/x]} = \begin{cases}
        \Val{\left(\Me{y}{C}{r}\right)} = \textrm{FIX}(\Le{a}{\Val[]{C}}{\Val[\theta,y=a]{r}}) & \textrm{si } x = y\\
        \textrm{FIX}(\Le{a}{\Val[]{C}}{\Val[\theta,y=a]{r[t/x]}}) = \textrm{FIX}(\Le{a}{\Val[]{C}}{\Val[\theta,x=\Val{t},y=a]{r}}) & \textrm{si } x \neq y
    \end{cases}\]
    Si consideramos $\theta$ del primer caso como $\theta,x=\Val[\theta]{t}$,
    hecho que no varía el entorno de la evaluación ya que $x=y$ y luego aparece
    $y=a$, resulta en ambos casos $\textrm{FIX}(\Le{a}{\Val[]{C}}{\Val[\theta,x=\Val{t},y=a]{r}}) = \Val[\theta,x=\Val{t}]{\Me{y}{C}{r}}$.
    \end{case}
    \begin{case}{$\textrm{let } y : C = r \textrm{ in } s$.}
        TODO
    \end{case}
    \end{proof}
    \setcounter{case}{0}
    Luego se realiza inducción sobre la relación $\rightarrow$.
    \begin{proof}[Casos base:]
    \begin{case}{$p \odot q \rightarrow n$.}
    Por definición, $\Val{p \odot q} = \Val{p} \odot \Val{q}$ si
    $\Val{p}, \Val{q} \in \mathbb{N}$. Por hipótesis,
    $p \odot q \rightarrow n$. A su vez, $\Val{p} = p$,
    $\Val{q} = q$, $\Val{n} = n$. Luego
    $\Val{p} \odot \Val{q} = p \odot q = n = \Val{n}$, que vale.
    \end{case}

    \begin{case}{$0\; ?\; t : u \rightarrow t$.}
    Por definición, $\Val{0\; ?\; t : u} = \Val{t}$. Por hipótesis,
    $0\; ?\; t : u \rightarrow t$ y trivialmente resulta
    $\Val{t} = \Val{t}$.
    \end{case}

    \begin{case}{$1\; ?\; t : u \rightarrow u$.}
    Por definición, $\Val{1\; ?\; t : u} = \Val{u}$. Por hipótesis,
    $1\; ?\; t : u \rightarrow u$ y análogamente se observa que
    $\Val{u} = \Val{u}$.
    \end{case}

    \begin{case}{$\Me{x}{A}{t} \rightarrow t [\Me{x}{A}{t}/x]$.}
    Por definición, $\Val{\Me{x}{A}{t}} = \mathrm{FIX}(\Le{a}{\Val[]{A}}{\Val[\theta,x=a]{t}})$.
    A su vez, por definición de punto fijo,
    $\left(\Le{a}{\Val{A}}{\Val[\theta,x=a]{t}}\right) \mathrm{FIX}(\Le{a}{\Val[]{A}}{\Val[\theta,x=a]{t}}) = \mathrm{FIX}(\Le{a}{\Val[]{A}}{\Val[\theta,x=a]{t}})$.
    Aplicando resulta $\Val[\theta,x=\mathrm{FIX}(\Le{a}{\Val[]{A}}{\Val[\theta,x=a]{t}})]{t}$.
    Por lema e hipótesis, esto equivale a $\Val{t [\Me{x}{A}{t}/x]}$
    \end{case}

    \begin{case}{$\textrm{let } x : A = t \textrm{ in } u \rightarrow u[t/x]$.}
    Por definición, $\Val{\textrm{let } x : A = t \textrm{ in } u} = \Val[\theta,x=\Val{t}]{u}$.
    Por hipótesis, $\textrm{let } x : A = t \textrm{ in } u \rightarrow u[t/x]$,
    y $\Val{u[t/x]}$ equivale a $\Val[\theta,x=\Val{t}]{u}$ por lema.
    \end{case}

    \begin{case}{$\left(\Le{x}{A}{u}\right) t \rightarrow u[t/x]$.}
    Análogo al caso anterior.
    \end{case}
    \end{proof}
    \setcounter{case}{0}
    \begin{proof}[Reglas de congruencia:]
    \begin{case}{\AxiomC{$t \rightarrow u$}\UnaryInfC{$t\; v \rightarrow u\; v$}\DisplayProof}
    Por hipótesis, $\Val{t} = \Val{u}$. Luego $\Val{t} \Val{v} = \Val{u} \Val{v}$.
    \end{case}
    \begin{case}{\AxiomC{$t \rightarrow u$}\UnaryInfC{$v\; t \rightarrow v\; u$}\DisplayProof}
    Análogo al caso anterior.
    \end{case}
    \begin{case}{\AxiomC{$t \rightarrow u$}\UnaryInfC{$\Le{x}{A}{t} \rightarrow \Le{x}{A}{u}$}\DisplayProof}
    Por hipótesis, $\Val[\rho]{t} = \Val[\rho]{u}$. Por definición de semántica denotacional,
    \[\Val{\Le{x}{A}{t}} = \Le{a}{\Val[]{A}}{\Val[\theta,x=a]{t}}\]
    En particular con $\rho = \theta,x=a$ resulta
    \[\Le{a}{\Val[]{A}}{\Val[\theta,x=a]{t}} = \Le{a}{\Val[]{A}}{\Val[\theta,x=a]{u}}\]
    que es la semántica denotacional de $\Le{x}{A}{u}$.
    \end{case}
    \begin{case}{\AxiomC{$t \rightarrow u$}\UnaryInfC{$t \odot v \rightarrow u \odot v$}\DisplayProof}
    Por hipótesis, $\Val{t} = \Val{u}$. Luego si $\Val{t}, \Val{u} \in \mathbb{N}$ resulta
    \[\Val{t \odot v} = \Val{t} \odot \Val{v} = \Val{u} \odot \Val{v} = \Val{u \odot v}\]
    y si $\Val{t}, \Val{u} \notin \mathbb{N}$, vale $\Val{t \odot v} = \bot_{\mathbb{N}} = \Val{u \odot v}$.
    \end{case}
    \begin{case}{\AxiomC{$t \rightarrow u$}\UnaryInfC{$v \odot t \rightarrow v \odot u$}\DisplayProof}
    Análogo al caso anterior.
    \end{case}
    \begin{case}{\AxiomC{$s \rightarrow s'$}\UnaryInfC{$s\; ?\; t : u \rightarrow s'\; ?\; t : u $}\DisplayProof}
    Por hipótesis, $\Val{s} = \Val{s'}$. Analizando luego los casos,
    \[\Val{s\; ?\; t : u} = \begin{dlrcases}
        \Val{t} & \textrm{si } \Val{s} = \Val{s'} = 0\\
        \Val{u} & \textrm{si } \Val{s} = \Val{s'} \in \mathbb{N}^{*}\\
        \bot_{\mathbb{A}} & \textrm{si } \Val{s} = \Val{s'} = \bot_{\mathbb{N}} \textrm{ y A es el tipo del término}
    \end{dlrcases}   = \Val{s'\; ?\; t : u}\]
    \end{case}
    \begin{case}{\AxiomC{$t \rightarrow t'$}\UnaryInfC{$s\; ?\; t : u \rightarrow s\; ?\; t' : u $}\DisplayProof}
    Por hipótesis, $\Val{t} = \Val{t'}$. Analizando luego los casos,
    \[\Val{s\; ?\; t : u} = \begin{dlrcases}
        \Val{t} = \Val{t'} & \textrm{si } \Val{s} = 0\\
        \Val{u} & \textrm{si } \Val{s} \in \mathbb{N}^{*}\\
        \bot_{\mathbb{A}} & \textrm{si } \Val{s} = \bot_{\mathbb{N}} \textrm{ y A es el tipo del término}
    \end{dlrcases}   = \Val{s\; ?\; t' : u}\]
    \end{case}
    \begin{case}{\AxiomC{$u \rightarrow u'$}\UnaryInfC{$s\; ?\; t : u \rightarrow s\; ?\; t : u' $}\DisplayProof}
    Análogo al caso anterior.
    \end{case}
    \begin{case}{\AxiomC{$t \rightarrow u$}\UnaryInfC{$\Me{x}{A}{t} \rightarrow \Me{x}{A}{u}$}\DisplayProof}
    Por hipótesis, $\Val[\rho]{t} = \Val[\rho]{u}$. Por definición,
    \[\Val{\Me{x}{A}{t}} = \mathrm{FIX}(\Le{a}{\Val[]{A}}{\Val[\theta,x=a]{t}})\]
    En particular con $\rho = \theta,x=a$ resulta
    \[\mathrm{FIX}(\Le{a}{\Val[]{A}}{\Val[\theta,x=a]{t}}) = \mathrm{FIX}(\Le{a}{\Val[]{A}}{\Val[\theta,x=a]{u}})\]
    que es la semántica denotacional de $\Val{\Me{x}{A}{t}}$.
    \end{case}
    \begin{case}{\AxiomC{$t \rightarrow t'$}\UnaryInfC{$\textrm{let } x : A = t \textrm{ in } u \rightarrow \textrm{let } x : A = t' \textrm{ in } u$}\DisplayProof}
    Por hipótesis, $\Val[\rho]{t} = \Val[\rho]{t'}$. Por definición,
    \[\Val{\textrm{let } x : A = t \textrm{ in } u} = \Val[\theta,x=\Val{t}]{u}\]
    Luego $\Val[\theta,x=\Val{t}]{u} = \Val[\theta,x=\Val{t'}]{u}$,
    que es la semántica denotacional de $\textrm{let } x : A = t' \textrm{ in } u$.
    \end{case}
    \begin{case}{\AxiomC{$u \rightarrow u'$}\UnaryInfC{$\textrm{let } x : A = t \textrm{ in } u \rightarrow \textrm{let } x : A = t \textrm{ in } u'$}\DisplayProof}
    Por hipótesis, $\Val[\rho]{u} = \Val[\rho]{u'}$. Por definición,
    \[\Val{\textrm{let } x : A = t \textrm{ in } u} = \Val[\theta,x=\Val{t}]{u}\]
    En particular con $\rho = \theta,x=\Val{t}$ se tiene $\Val[\theta,x=\Val{t}]{u} = \Val[\theta,x=\Val{t}]{u'}$,
    que es la semántica denotacional de $\textrm{let } x : A = t \textrm{ in } u'$.
    \end{case}
    \end{proof}
\end{homeworkProblem}

\end{document}
