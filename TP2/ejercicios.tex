\documentclass{article}

\usepackage[spanish,american]{babel}
\usepackage{fontspec}

\usepackage{fancyhdr}
\usepackage{extramarks}
\usepackage{amsmath}
\usepackage{amsthm}
\usepackage{amssymb}
\usepackage{amsfonts}
\usepackage{tikz}
\usepackage[plain]{algorithm}
\usepackage{algpseudocode}
\usepackage{enumitem}
\usepackage{physics}
\usepackage{units}
\usepackage{comment}
\usepackage{mathtools}
\DeclarePairedDelimiter\floor{\lfloor}{\rfloor}
\DeclarePairedDelimiter\ceil{\lceil}{\rceil}

\usetikzlibrary{automata,positioning}

%
% Basic Document Settings
%

\topmargin=-0.45in
\evensidemargin=0in
\oddsidemargin=0in
\textwidth=6.5in
\textheight=9.0in
\headsep=0.25in

\linespread{1.1}

\pagestyle{fancy}
\lhead{\hmwkAuthorName}
\chead{\hmwkClass: \hmwkTitle}
\rhead{\firstxmark}
\lfoot{\lastxmark}
\cfoot{\thepage}

\renewcommand\headrulewidth{0.4pt}
\renewcommand\footrulewidth{0.4pt}

\setlength\parindent{0pt}

%
% Create Problem Sections
%

\newcommand{\enterProblemHeader}[1]{
    \nobreak\extramarks{}{El ejercicio \arabic{#1} continúa en la página siguiente\ldots}\nobreak{}
    \nobreak\extramarks{Ejercicio \arabic{#1} (cont.)}{El ejercicio \arabic{#1} continúa en la página siguiente\ldots}\nobreak{}
}

\newcommand{\exitProblemHeader}[1]{
    \nobreak\extramarks{Ejercicio \arabic{#1} (cont.)}{El ejercicio \arabic{#1} continúa en la página siguiente\ldots}\nobreak{}
    \stepcounter{#1}
    \nobreak\extramarks{Ejercicio \arabic{#1}}{}\nobreak{}
}

\setcounter{secnumdepth}{0}
\newcounter{partCounter}
\newcounter{homeworkProblemCounter}
\setcounter{homeworkProblemCounter}{1}
\nobreak\extramarks{Ejercicio \arabic{homeworkProblemCounter}}{}\nobreak{}

%
% Homework Problem Environment
%
% This environment takes an optional argument. When given, it will adjust the
% problem counter. This is useful for when the problems given for your
% assignment aren't sequential. See the last 3 problems of this template for an
% example.
%
\newenvironment{homeworkProblem}[1][-1]{
    \ifnum#1>0
        \setcounter{homeworkProblemCounter}{#1}
    \fi
    \section{Ejercicio \arabic{homeworkProblemCounter}}
    \setcounter{partCounter}{1}
    \enterProblemHeader{homeworkProblemCounter}
}{
    \exitProblemHeader{homeworkProblemCounter}
}

%
% Homework Details
%   - Title
%   - Due date
%   - Class
%   - Section/Time
%   - Instructor
%   - Author
%

\newcommand{\hmwkTitle}{Trabajo Práctico 2}
\newcommand{\hmwkDueDate}{21 de Octubre de 2016}
\newcommand{\hmwkClass}{Computación cuántica}
\newcommand{\hmwkClassLong}{Introducción a la computación cuántica y fundamentos de lenguajes de programación}
\newcommand{\hmwkClassInstructor}{Profesor Alejandro Díaz-Caro}
\newcommand{\hmwkAuthorName}{Emilio López}

%
% Title Page
%

\title{
    \vspace{2in}
    \textmd{\textbf{\hmwkClass\footnote{Formalmente ``\hmwkClassLong''}:\ \hmwkTitle}}\\
    \normalsize\vspace{0.1in}\small{Fecha\ de\ entrega:\ \hmwkDueDate}\\
    \vspace{0.1in}\large{\textit{\hmwkClassInstructor}}
    \vspace{3in}
}

\author{\textbf{\hmwkAuthorName}}
\date{}

\renewcommand{\part}[1]{\textbf{\large Parte \Alph{partCounter}}\stepcounter{partCounter}\\}

%
% Various Helper Commands
%

% Useful for algorithms
\newcommand{\alg}[1]{\textsc{\bfseries \footnotesize #1}}

% For derivatives
\newcommand{\deriv}[1]{\frac{\mathrm{d}}{\mathrm{d}x} (#1)}

% For partial derivatives
\newcommand{\pderiv}[2]{\frac{\partial}{\partial #1} (#2)}

% Integral dx
\newcommand{\dx}{\mathrm{d}x}

% Alias for the Solution section header
\newcommand{\solution}{\vspace{1em}\textbf{\large Solución}\vspace{.5em}}

% Probability commands: Expectation, Variance, Covariance, Bias
\newcommand{\E}{\mathrm{E}}
\newcommand{\Var}{\mathrm{Var}}
\newcommand{\Cov}{\mathrm{Cov}}
\newcommand{\Bias}{\mathrm{Bias}}

\begin{document}

\maketitle

\pagebreak

\begin{homeworkProblem}[57]
    Suppose $\{L_{l}\}$ and $\{M_{m}\}$ are two sets of measurement
    operators. Show that a measurement defined by the measurement
    operators $\{L_{l}\}$ followed by a measurement defined by the
    measurement operators $\{M_{m}\}$ is physically equivalent to a
    single measurement defined by measurement operators $\{N_{lm}\}$
    with the representation $N_{lm} \equiv M_{m}L_{l}$.

    \solution

    Sea $\ket{\phi}$ el estado inicial. Midiendo con $\{L_{l}\}$ y luego con $\{M_{m}\}$ resulta
    \[\begin{aligned}
    \ket{\phi} &\xrightarrow{L} \frac{L_l \ket{\phi}}{\sqrt{\expval{L_{l}^{\dagger}L_l}{\phi}}}
        = r L_{l} \ket{\phi} = r \ket{\psi} \\
    r\ket{\psi} &\xrightarrow{M} \frac{M_m r \ket{\psi}}{\sqrt{\expval{M_{m}^{\dagger} r^{2} M_m}{\psi}}}
        = \frac{M_m r \ket{\psi}}{r \sqrt{\expval{M_{m}^{\dagger} M_m}{\psi}}}
        = \frac{M_m L_l \ket{\phi}}{\sqrt{\expval{L_{l}^{\dagger} M_{m}^{\dagger} M_m L_l}{\phi}}}
        = \frac{M_m L_l \ket{\phi}}{\sqrt{\expval{{\left(M_{m} L_{l}\right)}^{\dagger} \left(M_m L_l\right)}{\phi}}}
    \end{aligned}\]
    Por lo tanto el estado final es equivalente al obtenido al medir con $N_{lm} = M_{m} L_{l}$.
\end{homeworkProblem}

\begin{homeworkProblem}[65]
    Express the states $\nicefrac{1}{\sqrt{2}} \; (\ket{0} + \ket{1})$ and
    $\nicefrac{1}{\sqrt{2}} \; (\ket{0} - \ket{1})$ in a basis in which they
    are not the same up to a relative phase shift.

    \solution

    Expresados sobre la base $\{ \ket{+}, \ket{-} \}$, dichos estados
    equivalen a $\ket{+}$ y $\ket{-}$ respectivamente, y es trivial
    observar que no son el mismo con una diferencia de fase, ya que
    $\nexists \; \theta : e^{i \theta} = 0$.

\end{homeworkProblem}

\begin{homeworkProblem}[74]
    Suppose a composite of systems $A$ and $B$ is in the state
    $\ket{a}\ket{b}$, where $\ket{a}$ is a pure state of system $A$,
    and $\ket{b}$ is a pure state of system $B$. Show that the reduced
    density operator of system $A$ alone is a pure state.

    \solution

    Sea $\rho^{AB} = \dyad{ab}$. Luego $\rho^{A} = \textrm{tr}_{B}\left(\dyad{ab}\right) = \textrm{tr}_{B}\left(\dyad{a} \otimes \dyad{b}\right) = \dyad{a} \textrm{tr}\left(\dyad{b}\right)$.{}
    Como $\ket{b}$ es un estado puro, resulta $\textrm{tr}\left(\dyad{b}\right) = 1$.
    Por lo tanto $\rho^{A} = \dyad{a}$ que también es un estado puro.
\end{homeworkProblem}

\begin{homeworkProblem}[75]
    For each of the four Bell states, find the reduced density operator
    for each qubit.

    \solution

    Recordemos los cuatro estados de Bell:

    \[\begin{split}\begin{aligned}
    \beta_{00} &= \nicefrac{1}{\sqrt{2}} \left( \ket{00} + \ket{11} \right) \\
    \beta_{01} &= \nicefrac{1}{\sqrt{2}} \left( \ket{01} + \ket{10} \right) \\
    \end{aligned}\end{split}
    \quad\quad
    \begin{split}\begin{aligned}
    \beta_{10} &= \nicefrac{1}{\sqrt{2}} \left( \ket{00} - \ket{11} \right) \\
    \beta_{11} &= \nicefrac{1}{\sqrt{2}} \left( \ket{01} - \ket{10} \right) \\
    \end{aligned}\end{split}\]

    \[\begin{split}\begin{aligned}
    \rho_{0}^{\beta_{00}} &= \nicefrac{1}{2}\; \textrm{tr}_{0}\left(\dyad{00}{00} + \dyad{00}{11} + \dyad{11}{00} + \dyad{11}{11}\right) \\
    &= \frac{\dyad{0} + \dyad{1}}{2} = \frac{I}{2} \\
    \rho_{1}^{\beta_{00}} &= \nicefrac{1}{2}\; \textrm{tr}_{1}\left(\dyad{00}{00} + \dyad{00}{11} + \dyad{11}{00} + \dyad{11}{11}\right) \\
    &= \frac{\dyad{0} + \dyad{1}}{2} = \frac{I}{2} \\
    \rho_{0}^{\beta_{01}} &= \nicefrac{1}{2}\; \textrm{tr}_{0}\left(\dyad{01}{01} + \dyad{01}{10} + \dyad{10}{01} + \dyad{10}{10}\right) \\
    &= \frac{\dyad{0} + \dyad{1}}{2} = \frac{I}{2} \\
    \rho_{1}^{\beta_{01}} &= \nicefrac{1}{2}\; \textrm{tr}_{1}\left(\dyad{01}{01} + \dyad{01}{10} + \dyad{10}{01} + \dyad{10}{10}\right) \\
    &= \frac{\dyad{0} + \dyad{1}}{2} = \frac{I}{2} \\
    \end{aligned}\end{split}
    \quad\quad
    \begin{split}\begin{aligned}
    \rho_{0}^{\beta_{10}} &= \nicefrac{1}{2}\; \textrm{tr}_{0}\left(\dyad{00}{00} - \dyad{00}{11} - \dyad{11}{00} + \dyad{11}{11}\right) \\
    &= \frac{\dyad{0} + \dyad{1}}{2} = \frac{I}{2} \\
    \rho_{1}^{\beta_{10}} &= \nicefrac{1}{2}\; \textrm{tr}_{1}\left(\dyad{00}{00} - \dyad{00}{11} - \dyad{11}{00} + \dyad{11}{11}\right) \\
    &= \frac{\dyad{0} + \dyad{1}}{2} = \frac{I}{2} \\
    \rho_{0}^{\beta_{11}} &= \nicefrac{1}{2}\; \textrm{tr}_{0}\left(\dyad{01}{01} - \dyad{01}{10} - \dyad{10}{01} + \dyad{10}{10}\right) \\
    &= \frac{\dyad{0} + \dyad{1}}{2} = \frac{I}{2} \\
    \rho_{1}^{\beta_{11}} &= \nicefrac{1}{2}\; \textrm{tr}_{1}\left(\dyad{01}{01} - \dyad{01}{10} - \dyad{10}{01} + \dyad{10}{10}\right) \\
    &= \frac{\dyad{0} + \dyad{1}}{2} = \frac{I}{2} \\
    \end{aligned}\end{split}\]
\end{homeworkProblem}

\begin{homeworkProblem}[12]
    Probar que para todo operador $A$, $A^{\dagger}A$ es positivo.

    \solution

    $A^{\dagger}A$ será positivo si resulta que $\forall \psi : \expval{A^{\dagger}A}{\psi} \geq 0$.
    Dado que $\expval{A^{\dagger}A}{\psi} = \braket{A \psi}$ y que $\braket{\phi} = \lVert \phi \rVert^{2}$
    se puede concluir luego que $A^{\dagger}A$ es positivo para cualquier operador $A$.
\end{homeworkProblem}

\begin{homeworkProblem}[14]
    Sea un sistema cuántico descripto por el conjunto de estados puros
    $\{(\nicefrac{1}{4}, \ket{0}); (\nicefrac{3}{4}, \ket{+})\}$
    y otro sistema cuántico descripto por el conjunto
    $\{(\nicefrac{3}{4}, \ket{1}); (\nicefrac{1}{4}, \ket{-})\}$

    \begin{enumerate}[label=(\alph*)]
        \item Dar la matriz densidad del sistema completo compuesto por
        ambos sistemas.
        \item Utilizando la medición $\{ \dyad{00}, \dyad{11}, \dyad{01}, \dyad{10} \}$,
        dar los posibles resultados de la medición con sus probabilidades
        y el estado final del sistema en cada caso.
    \end{enumerate}

    \solution

    \begin{enumerate}[label=(\alph*)]
        \item La matriz densidad del primer sistema será
        $\rho^{A} = \nicefrac{1}{4} \dyad{0} + \nicefrac{3}{4} \dyad{+}$
        y la del segundo sistema será
        $\rho^{A} = \nicefrac{3}{4} \dyad{1} + \nicefrac{1}{4} \dyad{-}$.
        Por postulado 4, luego el estado del sistema completo compuesto
        por ambos sistemas será
        \[\begin{aligned}
        \rho^{AB} &= \rho^{A} \otimes \rho^{B} \\
        &= \left( \nicefrac{1}{4} \dyad{0} + \nicefrac{3}{4} \dyad{+} \right) \otimes \left(\nicefrac{3}{4} \dyad{1} + \nicefrac{1}{4} \dyad{-}\right) \\
        &= \nicefrac{3}{16} \dyad{01} + \nicefrac{1}{16} \dyad{0-} + \nicefrac{9}{16} \dyad{+1} + \nicefrac{3}{16} \dyad{+-} \\
        \end{aligned}\]
        Para poder luego operar de forma más sencilla, expandiremos
        la expresión para tener todo en función de $\ket{0}$ y $\ket{1}$:
        \[\begin{aligned}
        \rho^{AB} &= \nicefrac{3}{16} \dyad{01} +
                     \nicefrac{1}{16} \left(\dyad{0} \otimes \dyad{-}\right) +
                     \nicefrac{9}{16} \dyad{+1} + \nicefrac{3}{16} \dyad{+-} \\
                  &= \nicefrac{3}{16} \dyad{01} +
                     \nicefrac{1}{16} \left(\dyad{0} \otimes \nicefrac{1}{2} \left(\ket{0} - \ket{1}\right)\left(\bra{0} - \bra{1}\right)\right) +
                     \nicefrac{9}{16} \dyad{+1} + \nicefrac{3}{16} \dyad{+-} \\
                  &= \nicefrac{3}{16} \dyad{01} +
                     \nicefrac{1}{32} \dyad{00} - \nicefrac{1}{32} \dyad{00}{01} - \nicefrac{1}{32} \dyad{01}{00} + \nicefrac{1}{32} \dyad{01} + \\
                  &\quad\;\nicefrac{9}{16} \dyad{+1} + \nicefrac{3}{16} \dyad{+-}\\
                  &= \nicefrac{3}{16} \dyad{01} +
                     \nicefrac{1}{32} \dyad{00} - \nicefrac{1}{32} \dyad{00}{01} - \nicefrac{1}{32} \dyad{01}{00} + \nicefrac{1}{32} \dyad{01} + \\
                  &\quad\;\nicefrac{9}{16} \nicefrac{1}{2} \left(\ket{0} + \ket{1}\right) \left(\bra{0} + \bra{1}\right) \otimes \dyad{1} + \nicefrac{3}{16} \dyad{+-} \\
                  &= \nicefrac{3}{16} \dyad{01} +
                     \nicefrac{1}{32} \dyad{00} - \nicefrac{1}{32} \dyad{00}{01} - \nicefrac{1}{32} \dyad{01}{00} + \nicefrac{1}{32} \dyad{01} + \\
                  &\quad\;\nicefrac{9}{32} \dyad{01} + \nicefrac{9}{32} \dyad{01}{11} + \nicefrac{9}{32} \dyad{11}{01} + \nicefrac{9}{32} \dyad{11} + \nicefrac{3}{16} \dyad{+-} \\
                  &= \nicefrac{3}{16} \dyad{01} +
                     \nicefrac{1}{32} \dyad{00} - \nicefrac{1}{32} \dyad{00}{01} - \nicefrac{1}{32} \dyad{01}{00} + \nicefrac{1}{32} \dyad{01} + \\
                  &\quad\;\nicefrac{9}{32} \dyad{01} + \nicefrac{9}{32} \dyad{01}{11} + \nicefrac{9}{32} \dyad{11}{01} + \nicefrac{9}{32} \dyad{11} + \\
                  &\quad\;\nicefrac{3}{16} \nicefrac{1}{2} \left(\ket{0} + \ket{1}\right) \left(\bra{0} + \bra{1}\right) \otimes \nicefrac{1}{2} \left(\ket{0} - \ket{1}\right) \left(\bra{0} - \bra{1}\right) \\
                  &= \nicefrac{3}{16} \dyad{01} +
                     \nicefrac{1}{32} \dyad{00} - \nicefrac{1}{32} \dyad{00}{01} - \nicefrac{1}{32} \dyad{01}{00} + \nicefrac{1}{32} \dyad{01} + \\
                  &\quad\;\nicefrac{9}{32} \dyad{01} + \nicefrac{9}{32} \dyad{01}{11} + \nicefrac{9}{32} \dyad{11}{01} + \nicefrac{9}{32} \dyad{11} + \\
                  &\quad\;\nicefrac{3}{64} \left(\dyad{0} + \dyad{0}{1} + \dyad{1}{0} + \dyad{1}\right) \otimes \left(\dyad{0} - \dyad{0}{1} - \dyad{1}{0} + \dyad{1}\right)\\
                  &= \nicefrac{3}{16} \dyad{01} +
                     \nicefrac{1}{32} \dyad{00} - \nicefrac{1}{32} \dyad{00}{01} - \nicefrac{1}{32} \dyad{01}{00} + \nicefrac{1}{32} \dyad{01} + \\
                  &\quad\;\nicefrac{9}{32} \dyad{01} + \nicefrac{9}{32} \dyad{01}{11} + \nicefrac{9}{32} \dyad{11}{01} + \nicefrac{9}{32} \dyad{11} + \nicefrac{3}{64} \dyad{00} -\\
                  &\quad\;\nicefrac{3}{64} \dyad{00}{01} - \nicefrac{3}{64} \dyad{01}{00} + \nicefrac{3}{64} \dyad{01} + \nicefrac{3}{64} \dyad{00}{10} - \nicefrac{3}{64} \dyad{00}{11} - \nicefrac{3}{64} \dyad{01}{10} +\\
                  &\quad\;\nicefrac{3}{64} \dyad{01}{11} + \nicefrac{3}{64} \dyad{10}{00} - \nicefrac{3}{64} \dyad{10}{01} - \nicefrac{3}{64} \dyad{11}{00} + \nicefrac{3}{64} \dyad{11}{01} + \nicefrac{3}{64} \dyad{10}{10} - \\
                  &\quad\;\nicefrac{3}{64} \dyad{10}{11} - \nicefrac{3}{64} \dyad{11}{10} + \nicefrac{3}{64} \dyad{11}\\
                  &= \nicefrac{5}{64} \dyad{00}{00} - \nicefrac{5}{64} \dyad{00}{01} + \nicefrac{3}{64} \dyad{00}{10} - \nicefrac{3}{64} \dyad{00}{11}-\\
                  &\quad\;\nicefrac{5}{64} \dyad{01}{00} + \nicefrac{35}{64} \dyad{01}{01} - \nicefrac{3}{64} \dyad{01}{10} + \nicefrac{21}{64} \dyad{01}{11} +\\
                  &\quad\;\nicefrac{3}{64} \dyad{10}{00} - \nicefrac{3}{64} \dyad{10}{01} + \nicefrac{3}{64} \dyad{10}{10} - \nicefrac{3}{64} \dyad{10}{11} - \\
                  &\quad\;\nicefrac{3}{64} \dyad{11}{00} + \nicefrac{21}{64} \dyad{11}{01} - \nicefrac{3}{64} \dyad{11}{10} + \nicefrac{21}{64} \dyad{11}{11}
        \end{aligned}\]

        \item El postulado 3 enuncia que la probabilidad de obtener un resultado $m$
        (es decir, que se aplique la matriz $M_{m}$ al medir) es
        $p(m) = \textrm{tr} \left(M_{m}^{\dagger} M_{m} \rho\right)$,
        donde $\rho$ es la matriz densidad del sistema. El sistema evoluciona
        al estado $p(m)^{-1} M_{m} \rho M_{m}^{\dagger}$ luego de la medición.

        Adicionalmente, es importante notar que $\bra{\phi}\ket{\psi} = 1 \iff \phi = \psi$ y
        $\bra{\phi}\ket{\psi} = 0 \iff \phi \neq \psi$, para $\phi, \psi$ vectores de una base ortonormal.
        Se usará este hecho para simplificar sustancialmente los cálculos.

        En particular, para obtener 0 será
        \[\begin{aligned}
        p(0) &= \textrm{tr} \left( \left(\dyad{00}\right)^{\dagger} \dyad{00} \rho^{AB} \right) \\
        &= \textrm{tr} \left( \dyad{00} \rho^{AB} \right) \\
        &= \textrm{tr} \left( \nicefrac{5}{64}\ket{00}\bra{00}\ket{00}\bra{00} -
                              \nicefrac{5}{64}\ket{00}\bra{00}\ket{00}\bra{01} +
                              \nicefrac{3}{64}\ket{00}\bra{00}\ket{00}\bra{10} -
                              \nicefrac{3}{64}\ket{00}\bra{00}\ket{00}\bra{11} \right) \\
        &= \textrm{tr} \left( \nicefrac{5}{64}\dyad{00}{00} -
                              \nicefrac{5}{64}\dyad{00}{01} +
                              \nicefrac{3}{64}\dyad{00}{10} -
                              \nicefrac{3}{64}\dyad{00}{11} \right) \\
        &= \nicefrac{5}{64}
        \end{aligned}\]
        y evolucionará a
        \[\begin{aligned}
        \rho^{AB} \xrightarrow{M_{0}} p(0)^{-1} \dyad{00} \rho^{AB} \left(\dyad{00}\right)^{\dagger} =
        \frac{64}{5} \dyad{00} \left(\frac{5}{64} \dyad{00}\right) \dyad{00} = \dyad{00}
        \end{aligned}\]

        Para obtener 1 será
        \[\begin{aligned}
        p(0) &= \textrm{tr} \left( \left(\dyad{11}\right)^{\dagger} \dyad{11} \rho^{AB} \right) \\
        &= \textrm{tr} \left( \dyad{11} \rho^{AB} \right) \\
        &= \textrm{tr} \left( -\nicefrac{3}{64}\ket{11}\bra{11}\ket{11}\bra{00} +
                              \nicefrac{21}{64}\ket{11}\bra{11}\ket{11}\bra{01} -
                              \nicefrac{3}{64}\ket{11}\bra{11}\ket{11}\bra{10} +
                              \nicefrac{21}{64}\ket{11}\bra{11}\ket{11}\bra{11} \right) \\
        &= \textrm{tr} \left( -\nicefrac{3}{64}\dyad{11}{00} +
                              \nicefrac{21}{64}\dyad{11}{01} -
                              \nicefrac{3}{64}\dyad{11}{10} +
                              \nicefrac{21}{64}\dyad{11}{11} \right) \\
        &= \nicefrac{21}{64}
        \end{aligned}\]
        y evolucionará a
        \[\begin{aligned}
        \rho^{AB} \xrightarrow{M_{1}} p(1)^{-1} \dyad{11} \rho^{AB} \left(\dyad{11}\right)^{\dagger} =
        \frac{64}{21} \dyad{11} \left(\frac{21}{64} \dyad{11}\right) \dyad{11} = \dyad{11}
        \end{aligned}\]

        Para obtener 2 será
        \[\begin{aligned}
        p(0) &= \textrm{tr} \left( \left(\dyad{01}\right)^{\dagger} \dyad{01} \rho^{AB} \right) \\
        &= \textrm{tr} \left( \dyad{01} \rho^{AB} \right) \\
        &= \textrm{tr} \left(-\nicefrac{5}{64}\ket{01}\bra{01}\ket{01}\bra{00} +
                              \nicefrac{35}{64}\ket{01}\bra{01}\ket{01}\bra{01} -
                              \nicefrac{3}{64}\ket{01}\bra{01}\ket{01}\bra{10} +
                              \nicefrac{21}{64}\ket{01}\bra{01}\ket{01}\bra{11} \right) \\
        &= \textrm{tr} \left(-\nicefrac{5}{64}\dyad{01}{00} +
                              \nicefrac{35}{64}\dyad{01}{01} -
                              \nicefrac{3}{64}\dyad{01}{10} +
                              \nicefrac{21}{64}\dyad{01}{11} \right) \\
        &= \nicefrac{35}{64}
        \end{aligned}\]
        y evolucionará a
        \[\begin{aligned}
        \rho^{AB} \xrightarrow{M_{2}} p(2)^{-1} \dyad{01} \rho^{AB} \left(\dyad{01}\right)^{\dagger} =
        \frac{64}{35} \dyad{01} \left(\frac{35}{64} \dyad{01}\right) \dyad{01} = \dyad{01}
        \end{aligned}\]

        Para obtener 3 será
        \[\begin{aligned}
        p(0) &= \textrm{tr} \left( \left(\dyad{10}\right)^{\dagger} \dyad{10} \rho^{AB} \right) \\
        &= \textrm{tr} \left( \dyad{10} \rho^{AB} \right) \\
        &= \textrm{tr} \left(-\nicefrac{3}{64}\ket{10}\bra{10}\ket{10}\bra{00} -
                              \nicefrac{3}{64}\ket{10}\bra{10}\ket{10}\bra{01} +
                              \nicefrac{3}{64}\ket{10}\bra{10}\ket{10}\bra{10} -
                              \nicefrac{3}{64}\ket{10}\bra{10}\ket{10}\bra{11} \right) \\
        &= \textrm{tr} \left(-\nicefrac{3}{64}\dyad{10}{00} -
                              \nicefrac{3}{64}\dyad{10}{01} +
                              \nicefrac{3}{64}\dyad{10}{10} -
                              \nicefrac{3}{64}\dyad{10}{11} \right) \\
        &= \nicefrac{3}{64}
        \end{aligned}\]
        y evolucionará a
        \[\begin{aligned}
        \rho^{AB} \xrightarrow{M_{3}} p(3)^{-1} \dyad{10} \rho^{AB} \left(\dyad{10}\right)^{\dagger} =
        \frac{64}{3} \dyad{10} \left(\frac{3}{64} \dyad{10}\right) \dyad{10} = \dyad{10}
        \end{aligned}\]
    \end{enumerate}
\end{homeworkProblem}

\begin{homeworkProblem}[16]
    Utilizar la traza parcial para determinar cuales de los siguientes
    estados están enredados.

    \begin{enumerate}[label=(\alph*)]
        \item $\nicefrac{1}{\sqrt{2}} \; (\ket{0+} + \ket{1-})$
        \item $\nicefrac{1}{\sqrt{2}} \; (\ket{++} + \ket{--})$
        \item $\nicefrac{1}{\sqrt{3}} \; (\ket{++} + \ket{+-} + \ket{--})$
    \end{enumerate}

    \solution

    Dado que $\{\ket{0}, \ket{1}\}$ y $\{\ket{+}, \ket{-}\}$ son bases
    ortonormales de $\mathbb{C}^{2}$, podemos aplicar la descomposición
    de Schmidt a estos estados y obtener fácilmente los operadores densidad
    reducidos (Teorema 3.28 y Corolario 3.29 del apunte).

    \begin{enumerate}[label=(\alph*)]
        \item $\rho^{0} = \left(\nicefrac{1}{\sqrt{2}}\right)^{2} \; (\dyad{0} + \dyad{1})
                        = \nicefrac{1}{2} \dyad{0} + \nicefrac{1}{2} \dyad{1}$

        A su vez, $\textrm{tr}\left(\left(\rho^{0}\right)^{2}\right) = \left(\nicefrac{1}{2} \dyad{0} + \nicefrac{1}{2} \dyad{1}\right) \left(\nicefrac{1}{2} \dyad{0} + \nicefrac{1}{2} \dyad{1}\right)
            = \nicefrac{1}{4} \dyad{0} + \nicefrac{1}{4} \dyad{1} = \nicefrac{1}{2}$

        Como $\textrm{tr}\left(\left(\rho^{0}\right)^{2}\right) < 1$, se
        puede concluir que $\rho^{0}$ es un operador densidad
        correspondiente a un estado mixto. Por lo tanto, $\rho$ está
        enredado.

        \item $\rho^{0} = \left(\nicefrac{1}{\sqrt{2}}\right)^{2} \; (\dyad{+} + \dyad{-})
                        = \nicefrac{1}{2} \dyad{+} + \nicefrac{1}{2} \dyad{-}$

        Luego, $\textrm{tr}\left(\left(\rho^{0}\right)^{2}\right) = \left(\nicefrac{1}{2} \dyad{+} + \nicefrac{1}{2} \dyad{-}\right) \left(\nicefrac{1}{2} \dyad{+} + \nicefrac{1}{2} \dyad{-}\right)
            = \nicefrac{1}{4} \dyad{+} + \nicefrac{1}{4} \dyad{-} = \nicefrac{1}{2} < 1$

        Por lo tanto, $\rho$ está enredado.

        \item $\rho^{0} = \left(\nicefrac{1}{\sqrt{3}}\right)^{2} \; (\dyad{+} + \dyad{+} + \dyad{-})
                        = \nicefrac{2}{3} \dyad{+} + \nicefrac{1}{3} \dyad{-}$

        Luego, $\textrm{tr}\left(\left(\rho^{0}\right)^{2}\right) = \left(\nicefrac{2}{3} \dyad{+} + \nicefrac{1}{3} \dyad{-}\right) \left(\nicefrac{2}{3} \dyad{+} + \nicefrac{1}{3} \dyad{-}\right)
            = \nicefrac{4}{9} \dyad{+} + \nicefrac{1}{9} \dyad{-} = \nicefrac{5}{9} < 1$

        Por lo tanto, $\rho$ también está enredado en este caso.
    \end{enumerate}
\end{homeworkProblem}

\end{document}
