\documentclass{article}

\usepackage[spanish,american]{babel}
\usepackage{fontspec}

\usepackage{fancyhdr}
\usepackage{extramarks}
\usepackage{amsmath}
\usepackage{amsthm}
\usepackage{amsfonts}
\usepackage{tikz}
\usepackage[plain]{algorithm}
\usepackage{algpseudocode}
\usepackage{enumitem}
\usepackage{physics}
\usepackage{units}
\usepackage{comment}
\usepackage{mathtools}
\DeclarePairedDelimiter\floor{\lfloor}{\rfloor}
\DeclarePairedDelimiter\ceil{\lceil}{\rceil}

\usetikzlibrary{automata,positioning}

%
% Basic Document Settings
%

\topmargin=-0.45in
\evensidemargin=0in
\oddsidemargin=0in
\textwidth=6.5in
\textheight=9.0in
\headsep=0.25in

\linespread{1.1}

\pagestyle{fancy}
\lhead{\hmwkAuthorName}
\chead{\hmwkClass: \hmwkTitle}
\rhead{\firstxmark}
\lfoot{\lastxmark}
\cfoot{\thepage}

\renewcommand\headrulewidth{0.4pt}
\renewcommand\footrulewidth{0.4pt}

\setlength\parindent{0pt}

%
% Create Problem Sections
%

\newcommand{\enterProblemHeader}[1]{
    \nobreak\extramarks{}{El ejercicio \arabic{#1} continúa en la página siguiente\ldots}\nobreak{}
    \nobreak\extramarks{Ejercicio \arabic{#1} (cont.)}{El ejercicio \arabic{#1} continúa en la página siguiente\ldots}\nobreak{}
}

\newcommand{\exitProblemHeader}[1]{
    \nobreak\extramarks{Ejercicio \arabic{#1} (cont.)}{El ejercicio \arabic{#1} continúa en la página siguiente\ldots}\nobreak{}
    \stepcounter{#1}
    \nobreak\extramarks{Ejercicio \arabic{#1}}{}\nobreak{}
}

\setcounter{secnumdepth}{0}
\newcounter{partCounter}
\newcounter{homeworkProblemCounter}
\setcounter{homeworkProblemCounter}{1}
\nobreak\extramarks{Ejercicio \arabic{homeworkProblemCounter}}{}\nobreak{}

%
% Homework Problem Environment
%
% This environment takes an optional argument. When given, it will adjust the
% problem counter. This is useful for when the problems given for your
% assignment aren't sequential. See the last 3 problems of this template for an
% example.
%
\newenvironment{homeworkProblem}[1][-1]{
    \ifnum#1>0
        \setcounter{homeworkProblemCounter}{#1}
    \fi
    \section{Ejercicio \arabic{homeworkProblemCounter}}
    \setcounter{partCounter}{1}
    \enterProblemHeader{homeworkProblemCounter}
}{
    \exitProblemHeader{homeworkProblemCounter}
}

%
% Homework Details
%   - Title
%   - Due date
%   - Class
%   - Section/Time
%   - Instructor
%   - Author
%

\newcommand{\hmwkTitle}{Trabajo Práctico 2}
\newcommand{\hmwkDueDate}{21 de Octubre de 2016}
\newcommand{\hmwkClass}{Computación cuántica}
\newcommand{\hmwkClassLong}{Introducción a la computación cuántica y fundamentos de lenguajes de programación}
\newcommand{\hmwkClassInstructor}{Profesor Alejandro Díaz-Caro}
\newcommand{\hmwkAuthorName}{Emilio López}

%
% Title Page
%

\title{
    \vspace{2in}
    \textmd{\textbf{\hmwkClass\footnote{Formalmente ``\hmwkClassLong''}:\ \hmwkTitle}}\\
    \normalsize\vspace{0.1in}\small{Fecha\ de\ entrega:\ \hmwkDueDate}\\
    \vspace{0.1in}\large{\textit{\hmwkClassInstructor}}
    \vspace{3in}
}

\author{\textbf{\hmwkAuthorName}}
\date{}

\renewcommand{\part}[1]{\textbf{\large Parte \Alph{partCounter}}\stepcounter{partCounter}\\}

%
% Various Helper Commands
%

% Useful for algorithms
\newcommand{\alg}[1]{\textsc{\bfseries \footnotesize #1}}

% For derivatives
\newcommand{\deriv}[1]{\frac{\mathrm{d}}{\mathrm{d}x} (#1)}

% For partial derivatives
\newcommand{\pderiv}[2]{\frac{\partial}{\partial #1} (#2)}

% Integral dx
\newcommand{\dx}{\mathrm{d}x}

% Alias for the Solution section header
\newcommand{\solution}{\vspace{1em}\textbf{\large Solución}\vspace{.5em}}

% Probability commands: Expectation, Variance, Covariance, Bias
\newcommand{\E}{\mathrm{E}}
\newcommand{\Var}{\mathrm{Var}}
\newcommand{\Cov}{\mathrm{Cov}}
\newcommand{\Bias}{\mathrm{Bias}}

\begin{document}

\maketitle

\pagebreak

\begin{homeworkProblem}[57]
    Suppose $\{L_{l}\}$ and $\{M_{m}\}$ are two sets of measurement
    operators. Show that a measurement defined by the measurement
    operators $\{L_{l}\}$ followed by a measurement defined by the
    measurement operators $\{M_{m}\}$ is physically equivalent to a
    single measurement defined by measurement operators $\{N_{lm}\}$
    with the representation $N_{lm} \equiv M_{m}L_{l}$.

    \solution

    TODO
\end{homeworkProblem}

\begin{homeworkProblem}[65]
    Express the states $\nicefrac{1}{\sqrt{2}} \; (\ket{0} + \ket{1})$ and
    $\nicefrac{1}{\sqrt{2}} \; (\ket{0} - \ket{1})$ in a basis in which they
    are not the same up to a relative phase shift.

    \solution

    TODO
\end{homeworkProblem}

\begin{homeworkProblem}[74]
    Suppose a composite of systems $A$ and $B$ is in the state
    $\ket{a}\ket{b}$, where $\ket{a}$ is a pure state of system $A$,
    and $\ket{b}$ is a pure state of system $B$. Show that the reduced
    density operator of system $A$ alone is a pure state.

    \solution

    TODO
\end{homeworkProblem}

\begin{homeworkProblem}[75]
    For each of the four Bell states, find the reduced density operator
    for each qubit.

    \solution

    TODO
\end{homeworkProblem}

\begin{homeworkProblem}[12]
    Probar que para todo operador $A$, $A^{\dagger}A$ es positivo.

    \solution

    TODO
\end{homeworkProblem}

\begin{homeworkProblem}[14]
    Sea un sistema cuántico descripto por el conjunto de estados puros
    $\{(\nicefrac{1}{4}, \ket{0}); (\nicefrac{3}{4}, \ket{+})\}$
    y otro sistema cuántico descripto por el conjunto
    $\{(\nicefrac{3}{4}, \ket{1}); (\nicefrac{1}{4}, \ket{-})\}$

    \begin{enumerate}[label=(\alph*)]
        \item Dar la matriz densidad del sistema completo compuesto por
        ambos sistemas.
        \item Utilizando la medición $\{ \dyad{00}, \dyad{11}, \dyad{01}, \dyad{10} \}$,
        dar los posibles resultados de la medición con sus probabilidades
        y el estado final del sistema en cada caso.
    \end{enumerate}

    \solution

    TODO
\end{homeworkProblem}

\begin{homeworkProblem}[16]
    Utilizar la traza parcial para determinar cuales de los siguientes
    estados están enredados.

    \begin{enumerate}[label=(\alph*)]
        \item $\nicefrac{1}{\sqrt{2}} \; (\ket{0+} + \ket{1-})$
        \item $\nicefrac{1}{\sqrt{2}} \; (\ket{++} + \ket{--})$
        \item $\nicefrac{1}{\sqrt{3}} \; (\ket{++} + \ket{+-} + \ket{--})$
    \end{enumerate}

    \solution

    TODO
\end{homeworkProblem}

\end{document}
